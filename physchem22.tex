\documentclass{article}
\usepackage{amsmath}
\usepackage{geometry}
\geometry{a4paper, scale=0.8}
\begin{document}


\noindent \textbf{Q1:Prove the relation $\vec{S}^2\chi^{III}_S=2\hbar^2\chi^{III}_S$.}\\
\noindent $\vec{S}^2\chi^{III}_S=2\frac{1}{\sqrt{2}}\frac{3}{4}\hbar^2(\begin{pmatrix}1 & 0 \\ 0 & 1\end{pmatrix}_1\begin{pmatrix}1 \\ 0 \end{pmatrix}_1\begin{pmatrix}1 & 0 \\ 0 & 1\end{pmatrix}_2\begin{pmatrix}0 \\ 1 \end{pmatrix}_2+\begin{pmatrix}1 & 0 \\ 0 & 1\end{pmatrix}_2\begin{pmatrix}1 \\ 0 \end{pmatrix}_2\begin{pmatrix}1 & 0 \\ 0 & 1\end{pmatrix}_1\begin{pmatrix}0 \\ 1 \end{pmatrix}_1)+$\\
\noindent $2\frac{\hbar^2}{4}\frac{1}{\sqrt{2}}[\begin{pmatrix}0 & 1 \\ 1 & 0\end{pmatrix}_1\begin{pmatrix}1 \\ 0 \end{pmatrix}_1\begin{pmatrix}0 & 1 \\ 1 & 0\end{pmatrix}_2\begin{pmatrix}0 \\ 1 \end{pmatrix}_2+\begin{pmatrix}0 & 1 \\ 1 & 0\end{pmatrix}_2\begin{pmatrix}1 \\ 0 \end{pmatrix}_2\begin{pmatrix}0 & 1 \\ 1 & 0\end{pmatrix}_1\begin{pmatrix}0 \\ 1 \end{pmatrix}_1+$\\
\noindent $\begin{pmatrix}0 & -i \\ i & 0\end{pmatrix}_1\begin{pmatrix}1 \\ 0 \end{pmatrix}_1\begin{pmatrix}0 & -i \\ i & 0\end{pmatrix}_2\begin{pmatrix}0 \\ 1 \end{pmatrix}_2+\begin{pmatrix}0 & -i \\ i & 0\end{pmatrix}_2\begin{pmatrix}1 \\ 0 \end{pmatrix}_2\begin{pmatrix}0 & -i \\ i & 0\end{pmatrix}_1\begin{pmatrix}0 \\ 1 \end{pmatrix}_1+$\\
\noindent $\begin{pmatrix}1 & 0 \\ 0 & -1\end{pmatrix}_1\begin{pmatrix}1 \\ 0 \end{pmatrix}_1\begin{pmatrix}1 & 0 \\ 0 & -1\end{pmatrix}_2\begin{pmatrix}0 \\ 1 \end{pmatrix}_2+\begin{pmatrix}1 & 0 \\ 0 & -1\end{pmatrix}_2\begin{pmatrix}1 \\ 0 \end{pmatrix}_2\begin{pmatrix}1 & 0 \\ 0 & -1\end{pmatrix}_1\begin{pmatrix}0 \\ 1 \end{pmatrix}_1]$\\
\noindent =$2\hbar^2\frac{1}{\sqrt{2}}[\alpha(1)\beta(2)+\alpha(2)\beta(1)]$\\
\noindent =$2\hbar^2\chi^{III}_S$


\newpage
\noindent \textbf{Q2:Prove the relation $\vec{S}^2=S(S+1)\hbar^2$.}\\
\noindent Let $\phi(\vec{S}^2 S^{'}_z)$ be a state function for the system with eigenvalue $S^{'}_z$ for $\vec{S}_z$.\\
\noindent From $\vec{S}^2$ = $\vec{S}^2_x+\vec{S}^2_y+\vec{S}^2_z$ is obtained,\\
\noindent $\vec{S}^2_x+\vec{S}^2_y=\vec{S}^2-\vec{S}^2_z$\\
\noindent Applying both sides of the above equation to $\phi(\vec{S}^2 S^{'}_z)$,\\
\noindent $(\vec{S}^2_x+\vec{S}^2_y)\phi(\vec{S}^2 S^{'}_z)=(\vec{S}^2-S^{'2}_z)\phi(\vec{S}^2 S^{'}_z)$\\
\noindent Since $S_x$ and $S_y$ are real observables, $\vec{S}^2-S^{'2}_z$ is not negative and $\lvert S^{'}_z\rvert\leq\sqrt{\vec{S}^2}$. Thus $S^{'}_z$ has an upper and lower bound.\\
\noindent Two of the commutation relations for the components of $\vec{S}$ are,\\
\noindent [$\vec{S}_y$, $\vec{S}_z$]=${i}\hbar\vec{S}_x$ \\
\noindent [$\vec{S}_z$, $\vec{S}_x$]=${i}\hbar\vec{S}_y$\\
\noindent They can be combined to obtain two equations, which are written together using $\pm$ signs in the following,\\
\noindent $\vec{S}_z(\vec{S}_x\pm i\vec{S}_y)=(\vec{S}_x\pm i\vec{S}_y)(\vec{S}_z\pm \hbar)$,\\
\noindent where one of the equations uses the + sign and the other uses the - signs. Applying both sides of the above to $\phi(\vec{S}^2 S^{'}_z)$.\\
\noindent $\vec{S}_z(\vec{S}_x \pm i\vec{S}_y)\phi(\vec{S}^2 S^{'}_z)=(\vec{S}_x \pm i\vec{S}_y)(\vec{S}_z \pm \hbar)\phi(\vec{S}^2 S^{'}_z)$\\
\noindent $=(\vec{S}^{'}_z \pm \hbar)(\vec{S}_x \pm i\vec{S}_y)\phi(\vec{S}^2 S^{'}_z)$\\
\noindent The above shows that $(\vec{S}_x \pm i\vec{S}_y)\phi(\vec{S}^2 S^{'}_z)$ are two eigenfunctions of $\vec{S}_z$ with respective eigenvalues $S^{'}_z \pm \hbar$, unless one of the functions is zero, in which case it is not an eigenfunction. For the functions that are not zero,\\
\noindent $\phi(\vec{S}^2 S^{'}_z\pm\hbar)=(\vec{S}_x \pm i\vec{S}_y)\phi(\vec{S}^2 S^{'}_z)$\\
\noindent Further eigenfunctions of $\vec{S}_z$ and corresponding eigenvalues can be found by repeatedly applying $(\vec{S}_x \pm i\vec{S}_y)$ as long as the magnitude of the resulting eigenvalue is $\leq\sqrt{\vec{S}^2}$ . Since the eigenvalues of $\vec{S}_z$ are bounded, let $S^{0}_z$ be the lowest eigenvalue and $S^{1}_z$ be the highest. Then\\
\noindent $(\vec{S}_x-i\vec{S}_y)\phi(\vec{S}^2 S^{0}_z)=0$\\
\noindent and\\
\noindent $(\vec{S}_x+i\vec{S}_y)\phi(\vec{S}^2 S^{1}_z)=0$\\
\noindent since there are no states where the eigenvalue of $\vec{S}_z$ is \textless $S^0_z$ or \textgreater $S^1_z$. By applying \\
\noindent $(\vec{S}_x+i\vec{S}_y)$ to the first equation, $(\vec{S}_x-i\vec{S}_y)$ to the second, and using $\vec{S}^2_x+\vec{S}^2_y=\vec{S}^2-\vec{S}^2_z$, it can be shown that\\
\noindent $\vec{S}^2-(\vec{S}^0_z)^2-\hbar\vec{S}^0_z=0$\\
\noindent and\\
\noindent $\vec{S}^2-(\vec{S}^1_z)^2-\hbar\vec{S}^1_z=0$\\
\noindent Subtracting the first equation from the second and rearranging,\\
\noindent $(S^1_z+S^0_z)(S^0_z-S^1_z-\hbar)=0$\\
\noindent Since $S^1_z\geq S^0_z$, the second factor is negative. Then the first factor must be zero and thus $S^0_z=-S^1_z$\\
\noindent The difference $S^1_z-S^0_z$ comes from successive application of $\vec{S}_x-i\vec{S}_y$ or $\vec{S}_x+i\vec{S}_y$ which lower or raise the eigenvalue of $\vec{S}_z$ by $\hbar$ so that,\\
\noindent $\vec{S}_x-\vec{S}_y=2s\hbar$\\
\noindent where\\
\noindent s=0,$\frac{1}{2}$,1,$\frac{3}{2}$,...\\
\noindent Then using $S^0_z=-S^1_z$ and the above,\\
\noindent $S^0_z=-s\hbar$\\
\noindent Thus,\\
\noindent $\vec{S}^2=s(s+1)\hbar^2$


\newpage
\noindent \textbf{Q6:Prove the relation in F(x).}\\
\noindent $-\frac{4\pi}{V}\Sigma_{\lvert{k_2}\rvert<k_F}\frac{1}{\lvert{k_1-k_2}\rvert^2}$\\
\noindent $=-\frac{4\pi}{V}\frac{V}{(2\pi)^3}\int{dk_2}\frac{1}{\lvert{k_1-k_2}\rvert^2}$\\
\noindent $=-\frac{1}{2\pi^2}2\pi\int^{k_F}_{0}k^2_2dk_2\int^1_{-1}d\mu\frac{1}{k^2_1+k^2_2-2k_1k_2\mu}$\\
\noindent $=-\frac{1}{\pi}\frac{1}{k_1}\int^{k_F}_{0}k_2\ln\lvert\frac{k_1+k_2}{k_1-k_2}\rvert dk_2$\\
\noindent $=-\frac{1}{\pi}\frac{1}{k_1}[k_1k_2-\frac{1}{2}(k_1^2-k_2^2)\ln\lvert\frac{k_1+k_2}{k_1-k_2}\rvert]^{k_F}_{0}$\\
\noindent $=-\frac{k_F}{\pi}[1-\frac{1}{2}\frac{\frac{k^2_1}{k^2_F}-1}{\frac{k1}{k_F}}\ln\lvert\frac{\frac{k_1}{k_F}+1}{\frac{k_1}{k_F}-1}\rvert]$\\
\noindent $=-\frac{2k_F}{\pi}[\frac{1}{2}+\frac{1}{4}\frac{1-\frac{k^2_1}{k^2_F}}{\frac{k1}{k_F}}\ln\lvert\frac{\frac{k_1}{k_F}+1}{\frac{k_1}{k_F}-1}\rvert]$\\
\noindent Thus we get the relationship in F(x)


\newpage
\noindent \textbf{Q7:Prove the relation $\int\psi^*_\lambda(\textbf{r}_i)\psi^*_\mu(\textbf{r}_i)\frac{1}{r_{ij}}\psi_\lambda(\textbf{r}_i)\psi_\mu(\textbf{r}_i)d\textbf{r}_id\textbf{r}_j=\frac{1}{V}\frac{4\pi}{\lvert\textbf{k}_1-\textbf{k}_2\rvert^2}$.}\\
\noindent $\int\psi^*_\lambda(\textbf{r}_i)\psi^*_\mu(\textbf{r}_i)\frac{1}{r_{ij}}\psi_\lambda(\textbf{r}_i)\psi_\mu(\textbf{r}_i)d\textbf{r}_id\textbf{r}_j=\int e^{-ik_1r_i}e^{-ik_2r_j}\frac{1}{r_{ij}}\frac{1}{V}e^{ik_1r_i}e^{ik_2r_j}$\\
\noindent $=\frac{1}{V}\int e^{-i(k_1-k_2)(r_i-r_j)}\frac{1}{r_{ij}}dr_idr_j$\\
\noindent $=\frac{1}{V}\frac{4\pi}{\lvert k_1-k_2\rvert^2}$


\newpage
\noindent \textbf{Q8:Explain Koopman's theorem.}\\
\noindent Koopmans' theorem states that the first ionization energy of a molecule is equal to the negative of the energy of the highest occupied molecular orbital.\\
\noindent Koopmans' theorem uses the Hartree-Fock method for approximation of orbital energy $\varepsilon_i$ which is derived from the wavefunction of the spin orbital and the kinetic and nuclear attraction energies. This theorem applies when an electron is removed from a molecular orbital in order to form a positive ion. It was originally only used for ionization energies in a closed-shell system, but has been generalized to be used to calculate energy changes when electrons are added to or removed from a system. Based on this generalization, it is possible to use the same method to approximate the electron affinity. In this case, the molecular orbital energy would be the one associated with the orbital to which the electron is being added.


\newpage
\noindent \textbf{Q9:Discuss Hartree-Fock-Roothaan approximation.}\\
\noindent Let the molecular orbitals be expanded as $\phi_k{(x_1)}\Sigma^{K}_{n=1}c_{nk}\psi_n(x_1)$, here $\psi_n(x_1)$ is atom orbital or some other basis functions, $c_{nk}$ is coefficient. Put this into the HF equation $\hat{f}(x_1)\phi_k(x_1)=\varepsilon_k\phi_k(x_1)$, \\
\noindent $\hat{f}(x_1)\Sigma^{K}_{n=1}c_{nk}\psi_n(x_1)=\varepsilon_k\Sigma^{K}_{n=1}c_{nk}\psi_n(x_1)$\\
\noindent $\hat{f}$ is the Fock operator. Multiply both sides of the above to the left with $\psi^{*}_m(x_1)$ and then integral in space,\\
\noindent $FC_k=\varepsilon_kSC_k$\\
\noindent It solved the computational difficulty of the HF equation in polyatomic molecules.

\newpage
\noindent \textbf{Q10:Summarize super-exchange interaction.}\\
Super-exchange is the strong antiferromagnetic coupling between two next-to-nearest neighbour cations through a nonmagnetic anion.
In this way, it differs from direct exchange, in which there is coupling between nearest neighbor cations not involving an intermediary anion. 
Superexchange is a result of the electrons having come from the same donor atom and being coupled with the receiving ions' spins. 
If the two next-to-nearest neighbor positive ions are connected at 90 degrees to the bridging non-magnetic anion, then the interaction can be ferromagnetic interation.


\newpage
\noindent \textbf{Q11:Read the reference and summarize about double-exchange interaction.}\\
\noindent Double-exchange interaction occurs indirectly by means of spin coupling to mobile electrons which travel from one ion to the next. There are two levels for every spin arrangement of the ion cores, one as high in energy as the other is low.
\noindent Define J as intra-atomic exchange integral, b follows,\\
\noindent $(d_1\alpha|H|d_2\alpha)=b$\\
\noindent In the formula above, \\
\noindent $d_2\alpha^{'}:E=-JS$\\
\noindent $d_2\beta^{'}:E=+J(S+1)$\\
\noindent And\\
\noindent E$(d_2\alpha)=-JS$\\
\noindent E$(d_1\beta)=J(S+1)$\\
\noindent Qualitative conclusions could be drawn in two cases A(J$\gg$b) and B(J$\ll$b). \\
\noindent Case A, J$\gg$b, the transfer integral between any two ions depends on $\frac{cos \theta}{2}$ for the angle $\theta$ between their two spins. \\
\noindent Case B, J$\ll$b, in this case the free electrons will travel some distance while maintaining their spins without regard to be the directions of the ion core spins. Here the spin up and spin down electrons will have opposite energies. 


\newpage
\noindent \textbf{Q:Summarize the principle of synchrotron radiation.}\\
\noindent Synchroton radiation is the special case of charged particles moving at relativistic speed undergoing acceleration perpendicular to their direction of motion, typically in a magnetic field. In such a field, the force due to the field is always perpendicular to both the direction of motion and to the direction of field, as shown by the Lorentz force law.
\noindent The power carried by the radiation is found by the relativistic Larmor formula:\\
\noindent $P_\gamma=\frac{1}{6\pi\varepsilon}\frac{q^2a^2}{c^3}\gamma^4$\\
\noindent where\\
\noindent $\varepsilon_o$ is the vacuum permittivity\\
\noindent $q$ is the particle charge\\
\noindent $a$ is the magnitude of the acceleration\\
\noindent $c$ is the speed of light\\
\noindent $\gamma$ is the Lorentz factor\\
\noindent The synchrotron radiation should be,\\
\begin{itemize}
\item[*] High brilliance
\item[*] High level of polarization
\item[*] High collimation
\item[*] Low emittance
\item[*] Wide tunability in energy/wavelength by monochromatization
\item[*] Pulsed light emission
\end{itemize}



\newpage
\noindent \textbf{Q:Prove the Fermi's golden rule.}\\
\noindent Fermi's Golden rule\\
\noindent $I(\omega)\propto\Sigma_k\lvert\langle\Psi_n^{N-1}|a_k|\Psi_n^{N-1}\rangle\rvert^2\delta(\omega-(E_0^N-E_n^{N-1}+\mu))$\\
\noindent $\lvert\Psi_0^N\rangle$: initial state\\
\noindent $\lvert\Psi_0^{N-1}\rangle$: final state\\
\noindent a: dipole moment\\
\noindent The formula in the slide seems not correct because $\Psi_n^{N-1}$ appeared twice.
\noindent So I am going to prove another form of the Fermi's golden rule. \\
\noindent $H_0\lvert n\rangle=E_n\lvert n\rangle$\\
\noindent $\langle m|n\rangle=\delta_{mn}$\\
\noindent This equation means time independent Hamiltonian and energy of free particle system.\\
\noindent When Hamiltonian includes time dependent interactive part, Schrodinger equation becomes\\
\noindent $i\hbar\frac{\partial}{\partial t}\psi(t)=(H_0+H^{'}(t))\psi(t)$\\
\noindent $\lvert \psi(t)\rangle=\Sigma_nA_n(t)\lvert n\rangle$\\
\noindent Thus,\\
\noindent $i\hbar\frac{\partial}{\partial t}A_n(t)\lvert n\rangle=\Sigma_n(H_0+H^{'}(t))A_n(t)\lvert n\rangle$\\
\noindent $i\hbar\Sigma_n\frac{\partial}{\partial t}A_n(t)\langle m\lvert n\rangle=\langle m\lvert \Sigma_n(H_0+H^{'}(t))A_n(t)\rvert n\rangle$\\
\noindent Thus,\\
\noindent $\frac{\partial}{\partial t}A_m(t)=-\frac{i}{\hbar}E_mA_m(t)-\frac{i}{\hbar}\Sigma_nH^{'}_{mn}A_n(t)$\\
\noindent Here $H^{'}_{mn}=\langle m\lvert H^{'}(t)\rvert n\rangle$\\
\noindent Here we define\\
\noindent $A_m(t)=e^{-iE_mt/\hbar}a_m(t)$ $A_m(0)=a_m(0)$\\
\noindent Substituting the last two equations, and multiplying them by $e^{iE_mt/\hbar}$ from the left side,\\
\noindent $\frac{\partial}{\partial t}(e^{-iE_mt/\hbar}a_m(t))=-\frac{i}{\hbar}E_me^{-iE_mt/\hbar}a_m(t)-\frac{i}{\hbar}\Sigma_{n}H^{'}_{mn}(t)e^{-iE_nt/\hbar}a_n(t)$\\
\noindent $\frac{\partial}{\partial t}a_m(t)=-\frac{i}{\hbar}\Sigma_{n}e^{iE_mt/\hbar}H^{'}_{mn}(t)e^{-iE_nt/\hbar}a_n(t)$\\
\noindent Integrating the equation by $t^{'}$, \\
\noindent $\int^{t}_{0}dt^{'}\frac{\partial}{\partial t}a_m(t^{'})=-\frac{i}{\hbar}\Sigma_{n}\int^{t}_{0}e^{iE_mt^{'}/\hbar}H^{'}_{mn}(t^{'})e^{-iE_nt^{'}/\hbar}a_n(t^{'})$\\
\noindent $A_m(t)~e^{-iE_m t/\hbar}A_m(0)-\frac{i}{\hbar}\Sigma_n\int^{t}_{0}dt^{'}e^{-iE_m(t-t^{'})/\hbar}H^{'}_{mn}(t^{'})e^{-iE_n t^{'}/\hbar}A_n(0)$\\
\noindent Defining the initial state (t=0) as a simple form ($=\delta_{mn}$), and picking up only the second term (= interaction) in the right side,\\
\noindent $A_m(0)=\delta_{mn} (t=0)$ $m \neq n$ $H^{'}_{mn}=V_{mn}$\\
\noindent $A_m(t)=-\frac{i}{\hbar}V_{mn}e^{-iE_m t/\hbar}\int_0^{t}dt^{'}e^{i(E_m-E_n)t^{'}/\hbar}=-V_{mn}e^{-iE_m t/\hbar}\frac{e^{i(E_m-E_n)t/\hbar}-1}{E_m-E_n}$\\
\noindent Multiplying the equation by its complex conjugate, we get,\\
\noindent $\lvert A_m\rvert^2=\lvert V_{mn}\rvert^2\frac{2-e^{-i(E_m-E_n)t/\hbar}-e^{i(E_m-E_n)t/\hbar}}{(E_m-E_n)^2}$\\
\noindent Since,\\
\noindent $2-e^{-i\theta}-e^{i\theta}=4\sin^2\frac{\theta}{2}$\\
\noindent We have,\\
\noindent $\lvert A_m\rvert^2=4\frac{\lvert V_{mn}\rvert^2}{(E_m-E_n)^2}\sin^2\frac{(E_m-E_n)t/\hbar}{2}$\\
\noindent Delta function and its formula are,\\
\noindent $\delta(x)=\lim_{a\rightarrow\infty}\frac{\sin ax}{\pi x}$\\
\noindent $\int^{\infty}_{-\infty}dx\frac{\sin^2 x}{x^2}=\pi$\\
\noindent And we define $E_{mn}$, as follows,\\
\noindent $E_{mn}=\frac{E_m-E_n}{\hbar}$\\
\noindent $\frac{1}{\pi^2E^2_{mn}}\sin^2{\frac{E_{mn}t}{2}}=(\delta(E_{mn}))^2$\\
\noindent We use the following mathematical trick,\\
\noindent $\int^{\infty}_{-\infty}dE_{mn}(\delta(E_{mn}))^2=\frac{t}{2\pi}=\frac{t}{2\pi}\int^{\infty}_{-\infty}dE_{mn}\delta(E_{mn})$\\
\noindent $(\delta(E_{mn}))^2=\frac{t}{2\pi}\delta(E_{mn})$\\
\noindent From all of the above, we get,\\
\noindent $\lvert A_m\rvert^2=\frac{\lvert V_{mn}\rvert^2}{\hbar^2}2\pi t\delta(E_{mn})$\\
\noindent $=\frac{2\pi}{\hbar}\lvert V_{mn}\rvert^2t\delta(E_m-E_n)$\\
\noindent Move t to the left side,\\
\noindent $\omega_{mn}=\frac{2\pi}{\hbar}\lvert V_{mn}\rvert^2\delta(E_m-E_n)$\\
\noindent Here, $V_{mn}=$$\langle m\lvert H^{'}(t)\rvert n\rangle. $
\noindent This is the Fermi's golden rule.



\newpage
\noindent \textbf{Q:Prove $H_{SO}$ from Biot-Savart's law.}\\
\noindent The orbital motion of electrons rotating around the nucleus produces the orbital magnetic moment $-\mu_Bl$. The current generated by this positive charge creates a magnetic field at the position of the electron according to Biot-Savart's law.\\
\noindent $\vec{H}_{Ze}=\frac{Ze}{c}\frac{\vec{r}\times\vec{v}}{r^3}=\frac{Ze\hbar}{mc}\frac{1}{r^3}\vec{l}$\\
\noindent The Zeeman energy of electron spins by this magnetic field is calculated using $g=-g_S=-2$ of electrons.\\
\noindent $H_{SO}=-\vec{\mu}_S\cdot\vec{H}_{Ze}=-g\mu_B\vec{s}\cdot\vec{H}_{Ze}=\frac{Ze\hbar}{mc}\frac{1}{r^3}g_S\vec{s}\mu_B\vec{s}\cdot\vec{l}\rightarrow\frac{Z_{eff}}{2}(\frac{e\hbar}{mc})^2\frac{1}{r^3}\vec{s}\cdot\vec{l}$\\
\noindent $H_{SO}=\zeta\vec{s}\cdot\vec{l}$, ($\zeta>0$), $\zeta=\frac{Z_{eff}}{2}(\frac{e\hbar}{mc})^2\langle\frac{1}{r^3}\rangle$



\newpage
\noindent \textbf{Q:Prove $H_{SO}$ from Dirac equation.}\\
\noindent Spin-orbit coupling term couples spin of the electron $\sigma=2\frac{S}{\hbar}$ with movement of the electron m**v**=**p**-e**A** in presence of electrical field E.\\
\noindent $H_{SO}=-\frac{e\hbar}{4m^2c^2}\sigma\cdot[E\times(p-eA)]$\\
\noindent The maximal coupling is obtained when all three componets are perpendicular each other.\\
\noindent Spherical potential V(r)=V($\lvert r\rvert$)=V(r); static case $\frac{\partial}{\partial t}A=0$\\
\noindent $H_{SO}=-\frac{e\hbar}{4m^2c^2}\sigma\cdot[E\times p]$\\
\noindent $eE=-\bigtriangledown V(\lvert r\rvert)=\frac{dV(r)}{dr}\frac{r}{\lvert r\rvert}$\\
\noindent providing:\\
\noindent $ H_{SO}=\frac{\hbar}{4m^2c^2}\frac{1}{r}\frac{dV}{dr}\sigma\cdot(r\times p)=\frac{1}{2m^2c^2}\frac{1}{r}\frac{dV}{dr}s\cdot l=\zeta s\cdot l$




\newpage
\noindent \textbf{Q:Summarize the principle of photoemission spectroscopy.}\\
\noindent The physics behind photoemission spectroscopy is an application of the photoelectric effect. Photoelectron spectroscopy simply applies the photoelectric effect to free atoms or molecules instead of metals. In photoelectron spectroscopy, a sample is bombarded with high-energy radiation, which causes electrons to be ejected from the sample. The ejected electrons travel from the sample to an energy analyzer, where wheir kinetic energies are recorded, and then to a detector, which counts the number of photoelectrons at various kinetic energies.


\newpage
\noindent \textbf{Q:Explain the origin of chemical shift in XPS.}\\
\noindent Origin: Change in binding energy of a core electron of an element due to a change in the chemical bonding of that element.\\
\noindent Core binding energies are determined by:\\
\begin{itemize}
\item[*] electrostatic interaction between it and the nuclues, and reduced by:
\item[*] the electrostatic shielding of the nuclear charge from all other electrons in the atom 
\item[*] removal or addition of electronic charge as a result of changes in bonding will alter the shielding
\end{itemize}


\newpage
\noindent \textbf{Q:Explain the peak shift of Auger structure when changing photon energy.}\\
\noindent Two different X ray sources are used in XPS to distinguish Auger peaks from photo electron peaks. An Auger peak represents the kinetic energy of an auger electron which changes with the energy of primary X rays. Thus, auger peak will shift in apparent binding energy in XPS spectrum when X ray source is changed. [researchgate]\\


\newpage
\noindent \textbf{Q26: Summalize about Storner conditions.}\\
\noindent The Stoner condition is a condition to be fulfilled for the ferromagnetic order to arise in a simplified model of a solid. Ferromagnetism ultimately sterms from electron repulsion. The simplified model of a solid can be formulated in terms of dispersion relations for spin up and spin down electrons,\\
\noindent $E_{\uparrow}(k)=\varepsilon(k)-I\frac{N_{\uparrow}-N_{\downarrow}}{N}$\\
\noindent $E_{\downarrow}(k)=\varepsilon(k)+I\frac{N_{\uparrow}-N_{\downarrow}}{N}$\\
\noindent where the second term accounts for the exchange energy, $I$ is the Stoner parameter, $\frac{N_{\uparrow}}{N}$($\frac{N_{\downarrow}}{N}$) is the dimensionless density of spin up (down) electrons and $\varepsilon(k)$ is the dispersion relation of spinless electrons where the electron-electron interaction is disregarded. If $N_{\uparrow}+N_{\downarrow}$ is fixed, $E_{\uparrow}(k)$, $E_{\downarrow}(k)$ can be used to calculate the total energy of the system as a function of its polarization $P=\frac{N_{\uparrow}-N_{\downarrow}}{N}$. If the lowest total energy is found for P=0, the system prefers to remain paramagnetic but for larger values of I, polarized ground states occur.\\


\newpage
\noindent \textbf{Q: Explain the super-exchange interaction and Goodenough-Kanamori rule.}\\
\noindent - super-exchange interaction\\
\noindent Super-exchange interaction, is the strong (usually) antiferromagnetic coupling between two next-to-nearest neighbout cations through a non-magnetic anion. Superexchange is a result of the electrons having come from the same donor atom and being coupled with the receiving ions' spins. If the two next-to-nearest neighbor positive ions are connected at 90 degrees to the bridging non-magnetic anion, then the interaction can be a ferromagnetic interaction. [wiki]\\
\noindent - Goodenough-Kanamori rule\\
\noindent The Goodenough-Kanamori rule states that superexchange interactions are antiferromagnetic where the virtual electron transfer is between overlapping orbits that are each half-filled, but they are ferromagnetic where the virtual electron transfer is from a half-filled to an empty orbital or from a filled to a half-filled orbital. The Goodenough-Kanamori rule is the same for both superexchange and semicovalent exchange.\\
\noindent Where the two cation orbitals overlap the same p orbital of a shared anionas in a $180^{o}$ cation-anion-cation bridge, it is customary to introduce the virtual electron transfer from the shared anion to the interacting cations first as the covalent component of the cation orbital. The net spin of the cation orbital is not changed by addition of a covalent component, but the covalent component extends the cation wavefunction out over the anions to give an orbital overlap for the superexchange electron transfer. However, a pure semicovalent antiferromagnetic exchange can occur between two empty orbitals provided each cation carries a net spin and the empty orbitals share the same anion p orbital.\\
\noindent The theoretical basis for the Goodenough-Kanamori rule rests on four pillars,\\
\noindent - The spin angular momentum is conserved in an electron transfer, virtual or real\\
\noindent - The Pauli exclusion principal restricts electron transfer from a half-filled orbital or two-electron transfer from the same anion-p orbital.\\
\noindent - The intraatomic spin-spin potential exchange interaction is ferromagnetic and is determinative where the Pauli exclusion principle is not restrictive.\\
\noindent - Since spins are only oriented up ($\alpha$) or down ($\beta$) with respect to their spin axis, the angular dependence for an electron-spin transfer between cation spins having an angle $\theta$ between their spin axes is\\
\noindent  $\alpha_1=\alpha_2\cos{\theta/2}+\beta_2\cos{\theta/2}$\\
\noindent  $\beta_1=-\alpha_2\cos{\theta/2}+\beta_2\cos{\theta/2}$



\newpage
\noindent \textbf{Q: Explain the Tanabe-Sugano diagrams and how to use them for $d^n$.}\\
\noindent Tanabe-Sugamo diagrams are used in coordination chemistry to predict absorptions in the UV, visible and IR electromagnetic spectrum of coordination compounds. The x-axis of a Tanabe-Sugano diagram is expressed in terms of the ligand field splitting parameter, $\Delta$, or Dq, divided by the Racah parameter B. The y-axis is in terms of energy, E, also scaled by B. Three Racah parameters exist, A, B, and C, which describe various aspects of interelectronic repulsion. A is an average total interelectron repulsion. B and C correspond with individual d-electron repulsions. A is constant among d-electron configuration, and it is not necessary for calculating relative energies, hence its absence from Tanabe and Sugano's studies of complex ions. C is necessary only in certain cases. B is the most important for Racah's parameters in this case. One line corresponds to each electronic state. The bending of certain lines is due to the mixing of terms with the same symmetry. Although electronic transitions are only "allowed" if the spin multiplicity remains the same, energy levels for "spin-forbidden" electronic states are included in the diagrams. Each state is given its molecular-symmetry label, but "g" and "u" subscripts are usually left off because it is understood that all the states are gerade. Labels for each state are usually written on the right side of the table, though for more complicated diagrams labes may be written in other locations for clarity. Term symbols for a specific $d^n$ free ion are listed, in order of increasing energy, on the y-axis of the diagram. The relative order of energies is determined using Hund's rules. [wikis]\\
\noindent How to use them for $d^n$\\
\noindent \\
\noindent 1. Determine the d-configuration of the metal ion.\\
\noindent 2. Choose the appropriate Tanabe-SUgamo diagram matching the d-configuration\\
\noindent 3. Take a spectrum of the complex and identify $\lambda_{max}$ for spin-allowed (strong intensity) and spin forbidden (weak intensity) transitions.\\
\noindent 4. Convert $\lambda_{max}$ to wavenumbers and generate energy ratios to the lowest allowed transition\\
\noindent 5. Using a ruler, slide it across the printed diagram until the $\frac{E}{B}$ ratios between lines is equivalent to the ratios found in the last step.\\
\noindent 6. Solve for B using the $\frac{E}{B}$ values (y-axis) and $\frac{\Delta_{oct}}{B}$ to yield the ligand field splitting energy 10Dq.


\newpage
\noindent \textbf{Q: Explain the advantage of synchrotron radiation.}\\
\noindent I give the answer from the perspective of comparing to X-ray source.\\
\noindent \\
\noindent The most important advantage of synchrotron radiation over a laboratory X-ray source is its brilliance. [ESRF, France]\\
\noindent \\
\noindent Brilliance is a term that describes both the brightness and the angular spread of the beam. Higher brilliance lets us see more detail in the material under study.\\
\noindent \\
\noindent Other advantages,\\
\noindent \\
\noindent - high energy beams to penetrate deeper into matter\\
\noindent - small wavelengths permit the studying of tiny features\\
\noindent - synchrotron beams can be coherent and/or polarised, permitting specific experiments\\
\noindent - the synchrotron beam can be made to flash at a very high frequency, giving the light a time structure. This lets us follow chemical reactions on a very short time scale.


\newpage
\noindent \textbf{Q: Explain the EXAFS and XMCD.}\\
\noindent EXAFS,\\
\noindent Entended X-ray Absorption Fine Structure is a region of the spectrum obtained from XAS (X-ray Absorption Spectroscopy). Through mathematical analysis of this region, one can obtain local structural information for the atom in question.\\
\noindent XMCD,\\
\noindent X-ray magnetic circular dichroism (XMCD) is a difference spectrum of two X-ray absorption spectra (XAS) taken in a magnetic field, one taken with left circularly polarized light, and one with right circularly polarized light. [Coordination Chemistry Reviews, 2015]


\end{document}
